\documentclass[12pt]{article}
\usepackage[utf8]{inputenc}
\usepackage{amsmath}
\usepackage{amsfonts}
\usepackage{amssymb}
\usepackage{geometry}

\geometry{margin=1in}

\title{Homework 5: Lambda Calculus Workout}
\author{Gabe Giancarlo}
\date{\today}

\begin{document}

\maketitle

\section{Workout Problem}

Evaluate the following lambda calculus expression step by step:
$$(\lambda f.\lambda x.f(f(x))) (\lambda f.\lambda x.(f(f(f x))))$$

\section{Solution}

Let $M = \lambda f.\lambda x.f(f(x))$ and $N = \lambda f.\lambda x.(f(f(f x)))$.

We need to evaluate $M N$.

\begin{align}
M N &= (\lambda f.\lambda x.f(f(x))) (\lambda f.\lambda x.(f(f(f x)))) \\
&\rightsquigarrow \lambda x. (\lambda f.\lambda x.(f(f(f x)))) ((\lambda f.\lambda x.(f(f(f x)))) x) \\
&= \lambda x. (\lambda f.\lambda x.(f(f(f x)))) (f(f(f x))) \\
&\rightsquigarrow \lambda x. f(f(f(f(f(f x)))))
\end{align}

\section{Step-by-Step Explanation}

\begin{enumerate}
    \item \textbf{Initial expression:} $(\lambda f.\lambda x.f(f(x))) (\lambda f.\lambda x.(f(f(f x))))$
    
    \item \textbf{First β-reduction:} Apply the function $M = \lambda f.\lambda x.f(f(x))$ to the argument $N = \lambda f.\lambda x.(f(f(f x)))$.
    
    This substitutes $N$ for $f$ in $M$:
    $$\lambda x. N(N(x))$$
    
    \item \textbf{Expand $N$:} Replace $N$ with its definition:
    $$\lambda x. (\lambda f.\lambda x.(f(f(f x)))) ((\lambda f.\lambda x.(f(f(f x)))) x)$$
    
    \item \textbf{Second β-reduction:} Apply the inner function to $x$:
    $$(\lambda f.\lambda x.(f(f(f x)))) x \rightsquigarrow \lambda x.(f(f(f x)))$$
    
    But wait, this creates a variable capture issue. Let me be more careful.
    
    \item \textbf{Correct approach:} Let's rename variables to avoid capture:
    $$(\lambda f.\lambda x.(f(f(f x)))) x = (\lambda f.\lambda y.(f(f(f y)))) x \rightsquigarrow \lambda y.(f(f(f y)))$$
    
    \item \textbf{Final result:} The expression reduces to:
    $$\lambda x. f(f(f(f(f(f x)))))$$
\end{enumerate}

\section{Mathematical Interpretation}

This expression represents the composition of two functions:
\begin{itemize}
    \item The first function $M = \lambda f.\lambda x.f(f(x))$ applies its argument twice
    \item The second function $N = \lambda f.\lambda x.(f(f(f x)))$ applies its argument three times
    \item When composed, the result applies the argument six times total
\end{itemize}

This demonstrates the power of lambda calculus to represent complex computations through simple function composition and application, which was Church's original vision for a foundation of mathematics based purely on functions.

\end{document}
