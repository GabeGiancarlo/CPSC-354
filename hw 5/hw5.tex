\documentclass{article}

\usepackage{tikz} 
\usetikzlibrary{automata, positioning, arrows} 

\usepackage{amsthm}
\usepackage{amsfonts}
\usepackage{amsmath}
\usepackage{amssymb}
\usepackage{fullpage}
\usepackage{color}
\usepackage{parskip}
\usepackage{hyperref}
  \hypersetup{
    colorlinks = true,
    urlcolor = blue,       % color of external links using \href
    linkcolor= blue,       % color of internal links 
    citecolor= blue,       % color of links to bibliography
    filecolor= blue,        % color of file links
    }
    
\usepackage{listings}
\usepackage[utf8]{inputenc}                                                    
\usepackage[T1]{fontenc}                                                       

\definecolor{dkgreen}{rgb}{0,0.6,0}
\definecolor{gray}{rgb}{0.5,0.5,0.5}
\definecolor{mauve}{rgb}{0.58,0,0.82}

\lstset{frame=tb,
  language=haskell,
  aboveskip=3mm,
  belowskip=3mm,
  showstringspaces=false,
  columns=flexible,
  basicstyle={\small\ttfamily},
  numbers=none,
  numberstyle=\tiny\color{gray},
  keywordstyle=\color{blue},
  commentstyle=\color{dkgreen},
  stringstyle=\color{mauve},
  breaklines=true,
  breakatwhitespace=true,
  tabsize=3
}

\theoremstyle{plain} 
   \newtheorem{theorem}{Theorem}[section]
   \newtheorem{corollary}[theorem]{Corollary}
   \newtheorem{lemma}[theorem]{Lemma}
   \newtheorem{proposition}[theorem]{Proposition}
\theoremstyle{definition}
   \newtheorem{definition}[theorem]{Definition}
   \newtheorem{example}[theorem]{Example}
\theoremstyle{remark}    
  \newtheorem{remark}[theorem]{Remark}

\title{Lambda Calculus Workout}
\author{Gabriel Giancarlo \\ Chapman University}

\date{\today} 

\begin{document}

\maketitle

\begin{abstract}
This assignment provides practice with lambda calculus through function composition and beta-reduction exercises. We work through complex lambda expressions to develop fluency with the fundamental operations of functional programming.
\end{abstract}

\section{Workout Problem}

Evaluate the following lambda calculus expression step by step:
$$(\lambda f.\lambda x.f(f(x))) (\lambda f.\lambda x.(f(f(f x))))$$

\section{Solution}

Let $M = \lambda f.\lambda x.f(f(x))$ and $N = \lambda f.\lambda x.(f(f(f x)))$.

We need to evaluate $M N$.

\begin{align}
M N &= (\lambda f.\lambda x.f(f(x))) (\lambda f.\lambda x.(f(f(f x)))) \\
&\rightsquigarrow \lambda x. (\lambda f.\lambda x.(f(f(f x)))) ((\lambda f.\lambda x.(f(f(f x)))) x) \\
&= \lambda x. (\lambda f.\lambda x.(f(f(f x)))) (f(f(f x))) \\
&\rightsquigarrow \lambda x. f(f(f(f(f(f x)))))
\end{align}

\section{Step-by-Step Explanation}

\begin{enumerate}
    \item \textbf{Initial expression:} $(\lambda f.\lambda x.f(f(x))) (\lambda f.\lambda x.(f(f(f x))))$
    
    \item \textbf{First β-reduction:} Apply the function $M = \lambda f.\lambda x.f(f(x))$ to the argument $N = \lambda f.\lambda x.(f(f(f x)))$.
    
    This substitutes $N$ for $f$ in $M$:
    $$\lambda x. N(N(x))$$
    
    \item \textbf{Expand $N$:} Replace $N$ with its definition:
    $$\lambda x. (\lambda f.\lambda x.(f(f(f x)))) ((\lambda f.\lambda x.(f(f(f x)))) x)$$
    
    \item \textbf{Second β-reduction:} Apply the inner function to $x$:
    $$(\lambda f.\lambda x.(f(f(f x)))) x \rightsquigarrow \lambda x.(f(f(f x)))$$
    
    But wait, this creates a variable capture issue. Let me be more careful.
    
    \item \textbf{Correct approach:} Let's rename variables to avoid capture:
    $$(\lambda f.\lambda x.(f(f(f x)))) x = (\lambda f.\lambda y.(f(f(f y)))) x \rightsquigarrow \lambda y.(f(f(f y)))$$
    
    \item \textbf{Final result:} The expression reduces to:
    $$\lambda x. f(f(f(f(f(f x)))))$$
\end{enumerate}

\section{Mathematical Interpretation}

This expression represents the composition of two functions:
\begin{itemize}
    \item The first function $M = \lambda f.\lambda x.f(f(x))$ applies its argument twice
    \item The second function $N = \lambda f.\lambda x.(f(f(f x)))$ applies its argument three times
    \item When composed, the result applies the argument six times total
\end{itemize}

This demonstrates the power of lambda calculus to represent complex computations through simple function composition and application, which was Church's original vision for a foundation of mathematics based purely on functions.

\end{document}