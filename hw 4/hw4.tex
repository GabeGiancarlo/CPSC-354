\documentclass{article}

\usepackage{tikz} 
\usetikzlibrary{automata, positioning, arrows} 

\usepackage{amsthm}
\usepackage{amsfonts}
\usepackage{amsmath}
\usepackage{amssymb}
\usepackage{fullpage}
\usepackage{color}
\usepackage{parskip}
\usepackage{hyperref}
  \hypersetup{
    colorlinks = true,
    urlcolor = blue,       % color of external links using \href
    linkcolor= blue,       % color of internal links 
    citecolor= blue,       % color of links to bibliography
    filecolor= blue,        % color of file links
    }
    
\usepackage{listings}
\usepackage[utf8]{inputenc}                                                    
\usepackage[T1]{fontenc}                                                       

\definecolor{dkgreen}{rgb}{0,0.6,0}
\definecolor{gray}{rgb}{0.5,0.5,0.5}
\definecolor{mauve}{rgb}{0.58,0,0.82}

\lstset{frame=tb,
  language=haskell,
  aboveskip=3mm,
  belowskip=3mm,
  showstringspaces=false,
  columns=flexible,
  basicstyle={\small\ttfamily},
  numbers=none,
  numberstyle=\tiny\color{gray},
  keywordstyle=\color{blue},
  commentstyle=\color{dkgreen},
  stringstyle=\color{mauve},
  breaklines=true,
  breakatwhitespace=true,
  tabsize=3
}

\theoremstyle{plain} 
   \newtheorem{theorem}{Theorem}[section]
   \newtheorem{corollary}[theorem]{Corollary}
   \newtheorem{lemma}[theorem]{Lemma}
   \newtheorem{proposition}[theorem]{Proposition}
\theoremstyle{definition}
   \newtheorem{definition}[theorem]{Definition}
   \newtheorem{example}[theorem]{Example}
\theoremstyle{remark}    
  \newtheorem{remark}[theorem]{Remark}

\title{Termination}
\author{Gabriel Giancarlo \\ Chapman University}

\date{\today} 

\begin{document}

\maketitle

\begin{abstract}
This assignment explores termination analysis through measure functions. We examine two classic algorithms: the Euclidean algorithm for computing greatest common divisors and merge sort for array sorting. Both examples demonstrate how to construct appropriate measure functions to prove algorithm termination.
\end{abstract}

\section{Problem 4.1: Euclidean Algorithm}

Consider the following algorithm:

\begin{verbatim}
while b != 0:
    temp = b
    b = a mod b
    a = temp
return a
\end{verbatim}

Under certain conditions (which?) this algorithm always terminates.  

Find a measure function and prove termination.

\section{Problem 4.2: Merge Sort}

Consider the following fragment of an implementation of merge sort:

\begin{verbatim}
function merge_sort(arr, left, right):
    if left >= right:
        return
    mid = (left + right) / 2
    merge_sort(arr, left, mid)
    merge_sort(arr, mid+1, right)
    merge(arr, left, mid, right)
\end{verbatim}

Prove that
\[
    \varphi(left, right) = right - left + 1
\]
is a measure function for \texttt{merge\_sort}.

\section{Solution 4.1: Euclidean Algorithm}

The given algorithm is the \textbf{Euclidean Algorithm} for computing the greatest common divisor (gcd) of two integers.

\subsection{Conditions for Termination}
The algorithm requires that $a, b \in \mathbb{N}$ with $b \geq 0$. In particular:
\begin{itemize}
    \item $a$ and $b$ must be non-negative integers.
    \item If $b = 0$, the loop is skipped and the function returns $a$ immediately.
\end{itemize}

\subsection{Measure Function}
We define the measure function
\[
    \varphi(a,b) = b.
\]

\subsection{Proof of Termination}
At each iteration:
\[
    b \longmapsto a \bmod b,
\]
where $0 \leq a \bmod b < b$.

Thus $\varphi(a,b)$ strictly decreases whenever $b \neq 0$, and it always remains a non-negative integer. Since $\mathbb{N}$ is well-founded under $<$, infinite descent is impossible. Therefore the algorithm must terminate.

\section{Solution 4.2: Merge Sort}

We want to show that
\[
    \varphi(left, right) = right - left + 1
\]
is a measure function for \texttt{merge\_sort}.

\subsection{Non-negativity}
For all valid indices with $left \leq right$, we have
\[
    \varphi(left, right) = right - left + 1 \geq 1.
\]
Thus $\varphi$ always takes positive integer values.

\subsection{Decrease on Recursive Calls}
At each recursive step:
\[
    mid = \frac{left + right}{2}.
\]
The recursive calls are
\[
    \texttt{merge\_sort}(arr, left, mid), \qquad \texttt{merge\_sort}(arr, mid+1, right).
\]

Their measures are:
\[
    \varphi(left, mid) = mid - left + 1,
\]
\[
    \varphi(mid+1, right) = right - (mid+1) + 1 = right - mid.
\]

Since $left \leq mid < right$, both of these values are strictly smaller than
\[
    \varphi(left, right) = right - left + 1.
\]

\subsection{Termination}
Each recursive call strictly reduces the measure $\varphi$, which is bounded below by $1$. By well-foundedness of $\mathbb{N}$ under $<$, recursion cannot proceed indefinitely. Therefore the algorithm always terminates.

\section{Conclusion}

\begin{itemize}
    \item For the Euclidean algorithm, $\varphi(a,b)=b$ is a valid measure function, and the algorithm terminates for non-negative integer inputs with $b \geq 0$.
    \item For merge sort, $\varphi(left,right) = right - left + 1$ is a valid measure function, and recursion always terminates because the subproblems are strictly smaller.
\end{itemize}

\end{document}