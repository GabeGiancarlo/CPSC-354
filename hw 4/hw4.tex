\documentclass[12pt]{article}

% Packages
\usepackage{amsmath, amssymb, amsthm}
\usepackage{hyperref}
\usepackage{enumitem}
\usepackage{geometry}
\usepackage{mathtools}
\usepackage{microtype}

% Page setup
\geometry{margin=1in}

% Title info
\title{HW 4 --- PL 2025: Termination}
\author{Gabriel Giancarlo}
\date{\today}

\begin{document}

\maketitle

\section*{Problem 4.1}

Consider the following algorithm:

\begin{verbatim}
while b != 0:
    temp = b
    b = a mod b
    a = temp
return a
\end{verbatim}

Under certain conditions (which?) this algorithm always terminates.  

Find a measure function and prove termination.

\bigskip

\section*{Problem 4.2}

Consider the following fragment of an implementation of merge sort:

\begin{verbatim}
function merge_sort(arr, left, right):
    if left >= right:
        return
    mid = (left + right) / 2
    merge_sort(arr, left, mid)
    merge_sort(arr, mid+1, right)
    merge(arr, left, mid, right)
\end{verbatim}

Prove that
\[
    \varphi(left, right) = right - left + 1
\]
is a measure function for \texttt{merge\_sort}.

\newpage

\section*{Solution 4.1}

The given algorithm is the \textbf{Euclidean Algorithm} for computing the greatest common divisor (gcd) of two integers.

\subsection*{Conditions for Termination}
The algorithm requires that $a, b \in \mathbb{N}$ with $b \geq 0$. In particular:
\begin{itemize}
    \item $a$ and $b$ must be non-negative integers.
    \item If $b = 0$, the loop is skipped and the function returns $a$ immediately.
\end{itemize}

\subsection*{Measure Function}
We define the measure function
\[
    \varphi(a,b) = b.
\]

\subsection*{Proof of Termination}
At each iteration:
\[
    b \longmapsto a \bmod b,
\]
where $0 \leq a \bmod b < b$.

Thus $\varphi(a,b)$ strictly decreases whenever $b \neq 0$, and it always remains a non-negative integer. Since $\mathbb{N}$ is well-founded under $<$, infinite descent is impossible. Therefore the algorithm must terminate.

\bigskip

\section*{Solution 4.2}

We want to show that
\[
    \varphi(left, right) = right - left + 1
\]
is a measure function for \texttt{merge\_sort}.

\subsection*{Non-negativity}
For all valid indices with $left \leq right$, we have
\[
    \varphi(left, right) = right - left + 1 \geq 1.
\]
Thus $\varphi$ always takes positive integer values.

\subsection*{Decrease on Recursive Calls}
At each recursive step:
\[
    mid = \frac{left + right}{2}.
\]
The recursive calls are
\[
    \texttt{merge\_sort}(arr, left, mid), \qquad \texttt{merge\_sort}(arr, mid+1, right).
\]

Their measures are:
\[
    \varphi(left, mid) = mid - left + 1,
\]
\[
    \varphi(mid+1, right) = right - (mid+1) + 1 = right - mid.
\]

Since $left \leq mid < right$, both of these values are strictly smaller than
\[
    \varphi(left, right) = right - left + 1.
\]

\subsection*{Termination}
Each recursive call strictly reduces the measure $\varphi$, which is bounded below by $1$. By well-foundedness of $\mathbb{N}$ under $<$, recursion cannot proceed indefinitely. Therefore the algorithm always terminates.

\bigskip

\section*{Conclusion}

\begin{itemize}
    \item For the Euclidean algorithm, $\varphi(a,b)=b$ is a valid measure function, and the algorithm terminates for non-negative integer inputs with $b \geq 0$.
    \item For merge sort, $\varphi(left,right) = right - left + 1$ is a valid measure function, and recursion always terminates because the subproblems are strictly smaller.
\end{itemize}

\end{document}
