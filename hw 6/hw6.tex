\documentclass{article}

\usepackage{tikz} 
\usetikzlibrary{automata, positioning, arrows} 

\usepackage{amsthm}
\usepackage{amsfonts}
\usepackage{amsmath}
\usepackage{amssymb}
\usepackage{fullpage}
\usepackage{color}
\usepackage{parskip}
\usepackage{hyperref}
  \hypersetup{
    colorlinks = true,
    urlcolor = blue,       % color of external links using \href
    linkcolor= blue,       % color of internal links 
    citecolor= blue,       % color of links to bibliography
    filecolor= blue,        % color of file links
    }
    
\usepackage{listings}
\usepackage[utf8]{inputenc}                                                    
\usepackage[T1]{fontenc}                                                       

\definecolor{dkgreen}{rgb}{0,0.6,0}
\definecolor{gray}{rgb}{0.5,0.5,0.5}
\definecolor{mauve}{rgb}{0.58,0,0.82}

\lstset{frame=tb,
  language=haskell,
  aboveskip=3mm,
  belowskip=3mm,
  showstringspaces=false,
  columns=flexible,
  basicstyle={\small\ttfamily},
  numbers=none,
  numberstyle=\tiny\color{gray},
  keywordstyle=\color{blue},
  commentstyle=\color{dkgreen},
  stringstyle=\color{mauve},
  breaklines=true,
  breakatwhitespace=true,
  tabsize=3
}

\theoremstyle{plain} 
   \newtheorem{theorem}{Theorem}[section]
   \newtheorem{corollary}[theorem]{Corollary}
   \newtheorem{lemma}[theorem]{Lemma}
   \newtheorem{proposition}[theorem]{Proposition}
\theoremstyle{definition}
   \newtheorem{definition}[theorem]{Definition}
   \newtheorem{example}[theorem]{Example}
\theoremstyle{remark}    
  \newtheorem{remark}[theorem]{Remark}

\title{Advanced Lambda Calculus}
\author{Gabriel Giancarlo \\ Chapman University}

\date{\today} 

\begin{document}

\maketitle

\begin{abstract}
This assignment continues our exploration of lambda calculus with more advanced topics and complex function compositions. We build upon the foundations established in previous assignments to develop deeper understanding of functional programming principles.
\end{abstract}

\section{Advanced Lambda Calculus Exercises}

This week's assignment focuses on more complex lambda calculus expressions and their applications to functional programming.

\section{Exercise 1: Church Numerals}

Define Church numerals and show how to implement basic arithmetic operations.

\subsection{Church Numerals Definition}

Church numerals are a way of representing natural numbers using lambda calculus. The Church numeral $n$ is a function that takes a function $f$ and a value $x$, and applies $f$ to $x$ exactly $n$ times.

\begin{align}
0 &= \lambda f.\lambda x.x \\
1 &= \lambda f.\lambda x.f(x) \\
2 &= \lambda f.\lambda x.f(f(x)) \\
3 &= \lambda f.\lambda x.f(f(f(x)))
\end{align}

\subsection{Successor Function}

The successor function $S$ takes a Church numeral $n$ and returns $n+1$:

$$S = \lambda n.\lambda f.\lambda x.f(n f x)$$

\subsection{Addition}

Addition of Church numerals can be defined as:

$$+ = \lambda m.\lambda n.\lambda f.\lambda x.m f (n f x)$$

\section{Exercise 2: Boolean Operations}

Define Church booleans and show how to implement logical operations.

\subsection{Church Booleans}

\begin{align}
\text{true} &= \lambda x.\lambda y.x \\
\text{false} &= \lambda x.\lambda y.y
\end{align}

\subsection{Logical Operations}

\begin{align}
\text{and} &= \lambda p.\lambda q.p q p \\
\text{or} &= \lambda p.\lambda q.p p q \\
\text{not} &= \lambda p.\lambda x.\lambda y.p y x
\end{align}

\section{Exercise 3: Recursion and Fixed Points}

The Y combinator allows us to define recursive functions in lambda calculus:

$$Y = \lambda f.(\lambda x.f(x x))(\lambda x.f(x x))$$

\subsection{Example: Factorial}

We can define factorial using the Y combinator:

$$\text{factorial} = Y(\lambda f.\lambda n.\text{if } (n = 0) \text{ then } 1 \text{ else } n \times f(n-1))$$

\section{Conclusion}

These advanced lambda calculus concepts demonstrate the power and expressiveness of functional programming. The ability to represent numbers, booleans, and recursion purely through function application shows how lambda calculus can serve as a foundation for computation.

\end{document}